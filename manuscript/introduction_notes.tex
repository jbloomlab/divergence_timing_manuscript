\subsection*{Molecular clock dating \textit{underestimates} divergence dates for many different viruses.}

Underestimation might be extreme in RNA  viruses because of their mutation rate~\citep{holmes2003molecular}.

\textbf{\textit{Purifying Selection Can Obscure the Ancient Age of Viral Lineages}}\citep{wertheim2011purifying}
This paper uses simulations as well as sequences from Ebola, Measles, and Influenza to show how codon-based substitution models (specifically with site-to-site and lineage-to-lineage variation of non-synonymous to synonymous substitution rates.)
\begin{quote}
Statistical methods for molecular dating of viral origins have been used extensively to infer the time of most common recent ancestor for many rapidly evolving pathogens. However, there are a number of cases, in which epidemiological, historical, or genomic evidence suggests much older viral origins than those obtained via molecular dating. We demonstrate how pervasive purifying selection can mask the ancient origins of recently sampled pathogens, in part due to the inability of nucleotide-based substitution models to properly account for complex patterns of spatial and temporal variability in selective pressures. We use codon-based substitution models to infer the length of branches in viral phylogenies; these models produce estimates that are often considerably longer than those obtained with traditional nucleotide-based substitution models. Correcting the apparent underestimation of branch lengths suggests substantially older origins for measles, Ebola, and avian influenza viruses. This work helps to reconcile some of the inconsistencies between molecular dating and other types of evidence concerning the age of viral lineages.
\end{quote}

\subsubsection*{Actual examples}

There are many examples with real viruses sequences which show either a) that the current molecular clock method seems to drastically underestimate divergence times or b) substitution models which incorporate rate variation or purifying selection produces longer branches and therefore early divergence estimates. 

\textbf{HIV}  

HIV appears to be the best studied example of the molecular clock underestimation. 
While molecular clock methods estimate the common ancestor of HIV-1 and HIV-2 to be several thousand of years, the co-divergence theory of SIV and non-human primates would be the estimates around several million years ago. 

It seems that there is more and more evidence that the co-divergence hypothesis is probably not completely right but molecular clock methods are probably still underestimating the dates. 

\comment{There are more references for this section but I was working by way through the Wertheim bib.}
\comment{I also need to find the biogeography paper}

\textbf{\textit{A Challenge to the Ancient Origin of SIVagm
Based on African Green Monkey
Mitochondrial Genomes}}\citep{wertheim2007challenge}
This paper used more genomes of AGM to test the hypothesis that AGM and SIVagm co-diverged. They found that that mitochondrial AGM tree did not match the SIV tree. 
\begin{quote}
While the circumstances surrounding the origin and spread of HIV are becoming clearer, the particulars of the origin of simian immunodeficiency virus (SIV) are still unknown. Specifically, the age of SIV, whether it is an ancient or recent infection, has not been resolved. Although many instances of cross-species transmission of SIV have been documented, the similarity between the African green monkey (AGM) and SIVagm phylogenies has long been held as suggestive of ancient codivergence between SIVs and their primate hosts. Here, we present well-resolved phylogenies based on full- length AGM mitochondrial genomes and seven previously published SIVagm genomes; these allowed us to perform the first rigorous phylogenetic test to our knowledge of the hypothesis that SIVagm codiverged with the AGMs. Using the Shimodaira?H	asegawa test, we show that the AGM mitochondrial genomes and SIVagm did not evolve along the same topology. Furthermore, we demonstrate that the SIVagm topology can be explained by a pattern of west-to-east transmission of the virus across existing AGM geographic ranges. Using a relaxed molecular clock, we also provide a date for the most recent common ancestor of the AGMs at approximately 3 million years ago. This study substantially weakens the theory of ancient SIV infection followed by codivergence with its primate hosts.
\end{quote}

\textbf{\textit{Dating the Age of the SIV Lineages That Gave Rise to HIV-1 and HIV-2}}\citep{wertheim2009dating}
This paper showed that SIVsmm had an indistinguishable rate estimate to HIV-2 and that either SIV is very young or that both HIV and SIV dates are severally underestimated. 
\begin{quote}
Great strides have been made in understanding the evolutionary history of simian immunodeficiency virus (SIV) and the zoonoses that gave rise to HIV-1 and HIV-2. What remains unknown is how long these SIVs had been circulating in non- human primates before the transmissions to humans. Here, we use relaxed molecular clock dating techniques to estimate the time of most recent common ancestor for the SIVs infecting chimpanzees and sooty mangabeys, the reservoirs of HIV-1 and HIV-2, respectively. The date of the most recent common ancestor of SIV in chimpanzees is estimated to be 1492 (1266? 1685), and the date in sooty mangabeys is estimated to be 1809 (1729?1875). Notably, we demonstrate that SIV sequences sampled from sooty mangabeys possess sufficient clock-like signal to calibrate a molecular clock; despite the differences in host biology and viral dynamics, the rate of evolution of SIV in sooty mangabeys is indistinguishable from that of its human counterpart, HIV-2. We also estimate the ages of the HIV-2 human-to-human transmissible lineages and provide the first age estimate for HIV-1 group N at 1963 (1948?1977). Comparisons between the SIV most recent common ancestor dates and those of the HIV lineages suggest a difference on the order of only hundreds of years. Our results suggest either that SIV is a surprisingly young lentiviral lineage or that SIV and, perhaps, HIV dating estimates are seriously compromised by unaccounted-for biases.
\end{quote}

\textbf{\textit{Island Biogeography Reveals the Deep History of SIV}}\citep{worobey2010island}
This paper sequenced SIV from non-human primates which have been geographically isolated from humans for a long time. 
The SIV formed a monophyletic group on the tree with the SIV from the mainland which suggests that the virus had infected the most recent common ancestor of the species. 
This disagrees with molecular clock estimates. 

\textbf{Flu}

\comment{I know there are more flu papers, I just have not added them.}

\citep{wertheim2011purifying} looks at Eboloa, Measles, and IAV sequences. 
They are more concerned with \textit{dating} Ebola and Measles but show that the general patterns of purifying selection masking substitutions along long branches exists in IAV NA as well. 

\textbf{EBOV}
\citep{wertheim2011purifying} looks at Eboloa, Measles, and IAV sequences. 

\textbf{MeV}
\citep{wertheim2011purifying} looks at Eboloa, Measles, and IAV sequences. 

\textbf{Coronaviruses}

\textbf{\textit{A Case for the Ancient Origin of Coronaviruses}}\citep{wertheim2013case}
\begin{quote}
Coronaviruses are found in a diverse array of bat and bird species, which are believed to act as natural hosts. Molecular clock dating analyses of coronaviruses suggest that the most recent common ancestor of these viruses existed around 10,000 years ago. This relatively young age is in sharp contrast to the ancient evolutionary history of their putative natural hosts, which began diversifying tens of millions of years ago. Here, we attempted to resolve this discrepancy by applying more realistic evolutionary models that have previously revealed the ancient evolutionary history of other RNA viruses. By explicitly modeling variation in the strength of natural selection over time and thereby improving the modeling of substitution saturation, we found that the time to the most recent ancestor common for all coronaviruses is likely far greater (millions of years) than the previously inferred range.
\end{quote}

\textbf{HSV}

\textbf{\textit{Evolutionary Origins of Human Herpes Simplex Viruses 1 and 2}}\citep{wertheim2014evolutionary}
\begin{quote}
Herpesviruses have been infecting and co-diverging with their vertebrate hosts for hundreds of millions of years. The
primate simplex viruses exemplify this pattern of virus?host co-divergence, at a minimum, as far back as the most recent
common ancestor of New World monkeys, Old World monkeys, and apes. Humans are the only primate species known to
be infected with two distinct herpes simplex viruses: HSV-1 and HSV-2. Human herpes simplex viruses are ubiquitous,
with over two-thirds of the human population infected by at least one virus. Here, we investigated whether the additional
human simplex virus is the result of ancient viral lineage duplication or cross-species transmission. We found that
standard phylogenetic models of nucleotide substitution are inadequate for distinguishing among these competing
hypotheses; the extent of synonymous substitutions causes a substantial underestimation of the lengths of some of
the branches in the phylogeny, consistent with observations in other viruses (e.g., avian influenza, Ebola, and coronaviruses).
To more accurately estimate ancient viral divergence times, we applied a branch-site random effects likelihood
model of molecular evolution that allows the strength of natural selection to vary across both the viral phylogeny and the
gene alignment. This selection-informed model favored a scenario in which HSV-1 is the result of ancient co-divergence
and HSV-2 arose from a cross-species transmission event from the ancestor of modern chimpanzees to an extinct Homo
precursor of modern humans, around 1.6 Ma. These results provide a new framework for understanding human herpes
simplex virus evolution and demonstrate the importance of using selection-informed models of sequence evolution when
investigating viral origin hypotheses.
\end{quote}

\textbf{Herpes}

\textbf{Foamy virus}

\subsubsection*{TDRP - generally}

\subsection*{Substitution rate misspecification - generally}

\subsubsection*{Accounting for site-to-site variation, purifying selection}

\citep{halpern1998evolutionary}
\citep{brown1982mitochondrial}

\subsection*{other reasons why the divergence timing might be off.}
